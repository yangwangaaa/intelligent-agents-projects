%!TEX TS-program = xelatex
%!TEX encoding = UTF-8 Unicode
\documentclass[11pt,a4paper]{article}

%\usepackage[left=70pt,top=50pt,bottom=70pt,right=40pt]{geometry}
\usepackage{amsmath}
\usepackage{amsfonts}
\usepackage{amssymb}
\usepackage{fixltx2e}
\usepackage{cmap}
\usepackage{enumerate}
\usepackage{ifthen}
\usepackage{listings}
\usepackage{url}
\usepackage[T1]{fontenc}
\usepackage{fontspec}
\usepackage{xunicode}
\usepackage{xltxtra}
\setmainfont[Mapping=tex-text,Ligatures={Common,Rare,Discretionary}]{Linux Libertine O}
%\usepackage[scaled=.90]{helvet}
%\usepackage{courier}
%\usepackage{natbib}
%\renewcommand{\harvardurl}{URL: \url}
\usepackage{pdflscape}
\usepackage{qtree}
\usepackage{alltt}

\ifthenelse{\isundefined{\hypersetup}}{
    \usepackage[colorlinks=true,linkcolor=blue,urlcolor=blue]{hyperref}
    \urlstyle{same}
}{}

\hypersetup{
    pdftitle={Intelligent Agents - IN - Yoan Blanc, Tiziano Signo}
}
\title{\phantomsection%
    Implementing a first Application in RePast: \\
    A Rabbits Grass Simulation
}
\author{
    Yoan Blanc \texttt{<yoan.blanc@epfl.ch>}, 213552\\
    Tiziano Signo \texttt{<tiziano.signo@epfl.ch>}, 226511
}
\date{\today}


\begin{document}
\maketitle

\noindent
\begin{quote}{\it
    The goal of this exercise is to get you familiarized with \emph{RePast}, a Java
    agent-based simulation platform. During the Intelligent Agent course, you
    will work on a platform for simulating the Pickup and Delivery Problem,
    written on top of RePast, which is to be introduced in the next exercise.
    It is therefore important that you understand the underlying mechanisms by
    designing and implementing a simulation yourself.

    In this exercise, you should first go through the RePast documentation on
    their web page: \url{http://repast.sourceforge.net/repast_3/index.html}. Make
    sure that you work with Repast 3.1 (RepastJ) and not Repast Symphony! We
    recommend you to then read the tutorial described by John T. Murphy from
    the University of Arizona:
    \url{http://www.perfectknowledgedb.com/Tutorials/H2R/main.htm} and use this as a
    template to implementing the Rabbits Grass simulation.

    \medskip
    \emph{Setting up a directory structure}

    First create the root directory, for example CourseIntelligentAgents.
    Download rabbits.zip and unzip into this directory.

    \medskip
    \emph{The Rabbits Grass simulation}

    The Rabbits Grass simulation is a simulation of an ecosystem: rabbits
    wander around randomly on a discrete grid environment on which grass is
    growing randomly. When a rabbit bumps into some grass, it eats the grass
    and gains energy. If a rabbit gains enough energy, it reproduces. The
    reproduction takes some energy so the rabbit can not reproduce twice within
    the same simulation step. If the rabbit doesn't gain enough energy, it
    dies. The grass can be adjusted to grow at different rates and give the
    rabbits differing amounts of energy. It has to be possible to fully control
    the total amount of grass being grown at each simulation step. The model
    can be used to explore the competitive advantages of these variables.

    This model has been described at
    \url{http://ccl.northwestern.edu/netlogo/models/RabbitsGrassWeeds} for the
    NetLogo simulation toolkit. You final application should look like the
    following applet:
    \url{http://ccl.northwestern.edu/netlogo/models/run.cgi?RabbitsGrassWeeds.824.567},
    without the weeds.

    You have to program the Rabbits Grass Simulation in RePast, using the following requirements:

    \begin{description}
        \item[Grid]: the size of the world should be changeable. The default is
        a 20x20 grid. The world has no borders on the edges (thus, it is a
        torus).

        \item[Collisions]: different rabbits cannot stay on the same cell.

        \item[Legal moves]: only one-step moves to adjacent cells (north,
        south, east and west) are allowed.

        \item[Eat condition]: a rabbit can eat grass when it occupies the same
        cell.

        \item[Communication]: we assume that agents can not communicate with
        one another.

        \item[Visible range and directions]: all rabbits are blind and move
        randomly.

        \item[Creation]: at their births, rabbits are created at random places.

    \end{description}

    Implement sliders for the following variables of the simulation:

    \begin{itemize}
        \item \textbf{Grid size}
        \item The \textbf{number} of rabbits defines the initial number of
        rabbits

        \item The \textbf{birth threshold} of rabbits defines the energy level
        at which the rabbit reproduces.

        \item The \textbf{grass growth rate} controls the rate at which grass
        grows (total amount of grass added to the whole world within one simulation tick).
    \end{itemize}

    \textbf{Optional:} create a population plot to observe the evolution of the
    rabbits and the grass.
}\end{quote}

\newpage
\medskip
\emph{How the grass grows}

Grass is growing in patches having a certain amount of energy ($5$) which can be
accumulated over time until it reaches an upper bound ($15$). A each step, some
grass is spread randomly over the map and if an existing patch of grass already
exists, it'll simply increase its nutritive capabilities. Just like in real
life where grass grow bigger and more nutritious over time (until a limit).

\medskip
\emph{How the rabbits live and die}

A rabbit has a set of energy at its birth set randomly between $1$ and
\texttt{birthThreshold} so no newborn may directly bread. When a rabbit breads
it's losing half of its life.

Newborns are spawning randomly on the map and follow one direction (North,
South, East or West) unless they hit another rabbit or when a random condition
is met (e.g. with one chance in twenty).

Eating grabs the whole patch of grass at once leaving nothing behind. As
described before, some patches may be more nutritious than others if they
managed to slowly grow.

\medskip
\emph{Results}

The simulation seems to behave like the provided example. Some interesting
experiments are the ones where the population and the grass available
are oscillating between low population / high food available to the opposite
of high population and low food without being able to somehow stabilize. The
default values are not showing this because the grass is too nutricious (I
guess). Bigger maps are enabling this pattern to emerge.


\end{document}
